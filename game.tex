\documentclass{article}
\usepackage[utf8]{inputenc}
\usepackage[T2A]{fontenc}
\usepackage[english, russian]{babel}
\usepackage{amsmath}
\usepackage{amsfonts}
\usepackage{amssymb}
\usepackage{parskip}
\setlength{\parskip}{1em}
\setlength{\parindent}{0pt}

\begin{document}

\section{Концепция игры}
\subsection{Завязка}
Однажды в магическом мире, в далеких горах прошла дуэль между 
двумя могущественными магами. Их заклинания были настолько мощны, 
что после их битвы, в горах образовалась зона повышенного 
магического воздействия. Из-за больших всплесков магии начали 
образовываться аномалии --- шахты заполонили странные светящиеся 
кристаллы. Слухи об этом быстро распространились среди 
обитателей магического мира. Самые смелые авантюристы, 
жаждующие исследовать природу таинственных кристаллов, 
отправились в далекие горы, ведомые кристальной лихорадкой. 
Вскоре выяснилось, что обстановка крайне нестабильна --- 
магических частиц в горах накопилось столько, что вскоре 
произойдет неизбежный эфирный взрыв, который сотрёт всё 
до чего дотянется, поэтому всем нужно будет покинуть горы 
и вывезти кристаллы за день до взрыва. Теперь перед 
авантюристами стоит задача --- накопить как можно больше 
кристаллов ко дню отправки.
\subsection{Игра}
Кристальная лихорадка –-- это безкоалиционная неантагонистическая 
стратегическая игра для $N$-го числа игроков, на $T$ ходов, 
воплощающая конкурентную борьбу за ресурсы.
\subsubsection{Цель}
Цель каждого игрока получить как можно больше кристаллов 
ко дню отправки, тоесть накопить максимум очков к 
последнему ходу $T$.
\subsubsection{Игрок}
Каждый игрок владеет складом с кристаллами и набором юнитов.
\subsubsection{Юниты}
Рабочий --- каждый рабочий стабильно приносит +1 кристалл за ход.

Воин --- может либо атаковать вражеский склад, чтобы забрать кристаллы,
либо остаться защищать свой склад. Если воин стоит на охране, 
он может отбить атаку одного вражеского воина. 
Если воин атакует, и больше нет вражеских охранников, 
которые могли бы его остановить, то он атакует склад 
противника и крадёт из него 1 кристалл. 
Если в момент атаки склада, в нём 0 кристаллов, 
то воин разрушает склад, а игрок выбывает из игры. 
Если одного игрока атакует несколько игроков, то воины 
нападают по очереди по одному от каждого игрока в порядке 
номеров игроков, тоесть если на игрока $P_{3}$ напали 
2 воина игрока $P_{1}$ и 3 воина игрока $P_{2}$, то воины 
будут атаковать в порядке: $i_{p_{1}},i_{p_{2}},
i_{p_{1}},i_{p_{2}},i_{p_{2}}$, где $i_{pj}$ ---
воин $j$-того игрока.

Начальная стоимость каждого типа юнита равна 1. 
С каждой покупкой юнита определенного типа, цена на 
юнит этого типа возврастает на 1. Цена не сбрасывается 
на новом ходу, и сохраняется для последующих ходов.
\subsubsection{Старт}
Вначале игрокам даётся $S$ кристаллов и возможность купить 
стартовых юнитов, после чего начинается основная игра.
\subsubsection{Основная игра}
Игра состоит из последовательных ходов, в которых все игроки 
действуют синхронно. Каждый ход делится на три фазы:

1. Фаза приказов - игроки распределяют своих воинов, 
определяя команду для каждого воина - отправить его на 
конкретного врага, либо оставить на охране.

2. Фаза реализации - происходит выполнение всех приказов. 
Разрешаются все бои и начисляется добыча за них. 
Начисляются кристаллы за каждого рабочего.

3. Фаза найма - каждому игроку дана возможность приобрести 
юнитов. Игрок может как отказаться от покупки, так и купить 
любое число юнитов каждого типа, ограниченное только 
ресурсами игрока.
\end{document}