\documentclass{article}
\usepackage[utf8]{inputenc}
\usepackage[T2A]{fontenc}
\usepackage[english, russian]{babel}
\usepackage{amsmath}
\usepackage{amsfonts}
\usepackage{amssymb}
\usepackage{parskip}
\usepackage{algorithm}
\usepackage{algpseudocode}
\setlength{\parskip}{1em}
\setlength{\parindent}{0pt}

\begin{document}

\section{Концепция игры}
\subsection{Завязка}
Однажды в магическом мире, в далеких горах прошла дуэль между 
двумя могущественными магами. Их заклинания были настолько мощны, 
что после их битвы, в горах образовалась зона повышенного 
магического воздействия. Из-за больших всплесков магии начали 
образовываться аномалии --- шахты заполонили странные светящиеся 
кристаллы. Слухи об этом быстро распространились среди 
обитателей магического мира. Самые смелые авантюристы, 
жаждующие исследовать природу таинственных кристаллов, 
отправились в далекие горы, ведомые кристальной лихорадкой. 
Вскоре выяснилось, что обстановка крайне нестабильна --- 
магических частиц в горах накопилось столько, что вскоре 
произойдет неизбежный эфирный взрыв, который сотрёт всё 
до чего дотянется, поэтому всем нужно будет покинуть горы 
и вывезти кристаллы за день до взрыва. Теперь перед 
авантюристами стоит задача --- накопить как можно больше 
кристаллов ко дню отправки.
\subsection{Игра}
Кристальная лихорадка --- это безкоалиционная неантагонистическая 
стратегическая игра для $N$-го числа игроков, на $T$ ходов, 
воплощающая конкурентную борьбу за ресурсы.
\subsubsection{Цель}
Цель каждого игрока максимизировать количество кристаллов на складе
ко дню отправки, тоесть иметь максимум очков к 
последнему ходу $T$.
\subsubsection{Игрок}
Каждый игрок владеет складом с кристаллами и набором юнитов.
\subsubsection{Юниты}
Рабочий --- каждый рабочий стабильно приносит +1 кристалл за ход.

Воин --- может либо атаковать вражеский склад, чтобы забрать кристаллы,
либо остаться защищать свой склад. Если воин стоит на охране, 
он может отбить атаку одного вражеского воина. 
Если воин атакует, и больше нет вражеских охранников, 
которые могли бы его остановить, то он атакует склад 
противника и крадёт из него 1 кристалл. 
Если в момент атаки склада, в нём 0 кристаллов, 
то красть нечего, и воин не получает ничего. 
Если одного игрока атакует несколько игроков, то воины 
нападают по очереди по одному от каждого игрока в порядке 
номеров игроков, тоесть если на игрока $P_{3}$ напали 
2 воина игрока $P_{1}$ и 3 воина игрока $P_{2}$, то воины 
будут атаковать в порядке: $i_{p_{1}},i_{p_{2}},
i_{p_{1}},i_{p_{2}},i_{p_{2}}$, где $i_{pj}$ ---
воин $j$-того игрока.

Начальная стоимость каждого типа юнита равна 1. 
С каждой покупкой юнита определенного типа, цена на 
юнит этого типа возврастает на 1. Цена не сбрасывается 
на новом ходу, и сохраняется для последующих ходов.
\subsubsection{Старт}
Вначале игрокам даётся $S$ кристаллов и возможность купить 
стартовых юнитов, после чего начинается основная игра.
\subsubsection{Основная игра}
Игра состоит из последовательных ходов, в которых все игроки 
действуют синхронно. Каждый ход делится на три фазы:

1. Фаза приказов --- игроки распределяют своих воинов, 
определяя команду для каждого воина --- отправить его на 
конкретного врага, либо оставить на охране.

2. Фаза реализации --- происходит выполнение всех приказов. 
Начисляются кристаллы за каждого рабочего, разрешаются все бои и
начисляется добыча за них.

3. Фаза найма --- каждому игроку дана возможность приобрести 
юнитов. Игрок может как отказаться от покупки, так и купить 
любое число юнитов каждого типа, ограниченное только 
ресурсами игрока.

\section{Математическая формализация}
\subsection{Формализация игры}
\subsubsection{Основа}
В игре учавствует N игроков. Обозначим список игроков:
$$
P = \{1, 2, \dots, N\}, \quad N \geq 2
$$
Игра состоит из $T$ последовательных ходов. Конкретный ход обозначается
через $t$:
$$
t = 1, 2, \dots, T
$$
\subsubsection{Произвольный ход}
Рассмотрим игру в рамках одного произовльного хода t.

Состояние игрока $i$ на начало хода $t$ описывается вектором:
$$
state_i^t = (v_i^t, w_i^t, r_i^t)
$$
где:
\begin{itemize}
\item $v_i^t \in \mathbb{Z}_{\ge 0}$ --- количество кристаллов
\item $w_i^t \in \mathbb{Z}_{\ge 0}$ --- количество воинов
\item $r_i^t \in \mathbb{Z}_{\ge 0}$ --- количество рабочих
\end{itemize}

Тогда состояние игры на ходу t описывается вектором состояний
всех игроков на начало хода t:
$$
G^t=(state_1^t, state_2^t, \dots, state_N^t)
$$

1. Фаза приказов

Игрок распределяет своих воинов $w_i^t$:
$$
order^t_{i} = (a^t_{1},a^t_{2},\dots,a^t_{N}) \in D^t_{order,i}
$$
где:
\begin{itemize}
    \item $a_{j}^t \in \mathbb{Z}_{\ge 0}$ — количество воинов,
    отправленных для атаки игрока $j$, если $j \neq i$, или оставленных
    для защиты, если $j = i$
    \item Выполняется ограничение: $\sum_{j \in P} a_{j}^t = w_i^t$
\end{itemize}

2. Фаза реализации

Обозначим количество кристаллов у каждого игрока после фазы реализации
вектором:
$$
\tilde{v}^t = (\tilde{v}^t_1, \tilde{v}^t_2, \dots, \tilde{v}^t_N)
$$
Сначала начисляется добыча рабочих. Количество кристаллов для $i$-го
игрока будет:
$$
\tilde{v}^t_i = v^t_i + r^t_i
$$
После начинается разрешение боев. Обозначим приращение кристаллов
каждого игрока после боев, через вектор:
$$
loot^t = (l_1,l_2,\dots,l_N), \ \text{где } l_i \text{ --- приращение кристаллов i-го игрока}
$$
В начале фазы: $loot^t=(0,0,\dots,0)$.

Воины атакуют в порядке очереди, очередь можно обозначить как множество:
$$
Q = \{k_1, k_2, k_3, \dots\}, \ \text{где } k_i \text{ --- номер игрока, которому принадлежит воин}
$$
Обозначим $d^t_i=order^t_{i, i}$ - текущее число охраняющих воинов у $i$-го
игрока.
$$
\sum_{j \in P\setminus\{i\}}order^t_{j,i}=K \text{--- общее число атакующих воинов}
$$
Очередь формируется следующим образом:
\begin{itemize}
    \item Изначально $Q=\varnothing$
    \item До тех пор пока $|Q| \neq K$:
    для $k = 1, 2, \dots, N$ в порядке возрастания, где $k \neq i$:
    \begin{align*}
        \text{Если } order^t_{k,i} > 0 \text{, то } & order^t_{k,i} \leftarrow order^t_{k,i} - 1 \\
        & Q \leftarrow Q \cup \{k\}
    \end{align*}
\end{itemize}

Атака на $i$-го игрока одним воином $k$-го игрока описывается функцией:
$$
A(k, i, d, v, l) = \begin{cases}
    (d-1,v,l), \quad \text{if } d > 0 \\
    (d,v-1,l+1), \quad \text{if } d=0 \ \cap \ v>0 \\
    (d, v, l), \quad \text{else}
\end{cases}
$$
где $d$ --- свободные охранники $i$-того игрока, $v$ --- количество кристаллов
$i$-того игрока, $l$ --- приращение кристаллов $k$-того игрока.

Полная атака на $i$-го игрока это применение функции $A$ последовательно
к каждому элементу $Q$:
$$
\forall k \in Q: (d^t_i,v^t_i,loot^t_k) = A(k, i, d^t_i,v^t_i,loot^t_k)
$$

После этого начисляется добыча для каждого игрока:
$$
\tilde{\tilde{v}}^t=(\tilde{v}^t_1+loot^t_1, \tilde{v}^t_2+loot^t_2,\dots,
\tilde{v}^t_N+loot^t_N)
$$
3. Фаза найма

После каждой покупки юнита определенного типа, цена на этот тип юнита 
возрастает на 1. Тогда стоимость найма следующего юнита типа $c$ 
вычисляется по формуле $p_{c} = n_{c} + 1$, где $n_{c}$ --- 
текущее число юнитов типа $c$.

Стоимость найма $p_{c}^t(x)$ числа юнитов $x$ типа $c$ на ходу $t$ 
для игрока $i$ вычисляется по формуле:
$$
p_{c,i}^t(x) = \sum_{i=n_{c}^t+1}^{n_{c}^t + x}i \ , 
\ \text{где } n_{c}^t \ \text{- число юнитов типа } c \ 
\text{на ходе } t
$$
Итоговая стоимость найма $x$ воинов и $y$ рабочих 
за ход для игрока $i$:
$$
p_{i}^t(x,y)=p^t_{w}(x) + p_{r}^t(y)
$$
Игрок выбирает, сколько воинов и рабочих нанять. 
это пара неотрицательных целых чисел, удовлетворяющих условию, 
что затраты не превышают допустимые ресурсы:
$$
hire^t_{i}=(w,r) \in D^t_{hire,i} = \{ (x,y) \ | 
\ ,\ x\geq 0, \ y \geq 0,\ p_{i}^t(x,y) \leq \tilde{\tilde{v}}^t_{i}\}
$$

Далее у всех игроков обновляется количество кристаллов в зависимости от добычи:
$$
state_{i}^{t+1}= (\tilde{\tilde{v}}^t_{i} - p^t_{i}(hire^t_{i,1},hire^t_{i,2}),
w_i^t + hire^t_{i,1},r_i^t + hire^t_{i,2})
$$


\subsubsection{Стартовый ход}
В начале игры каждый игрок получает $S$ кристаллов. Начальное состояние всех игроков:
$$
\text{state}_i^0 = (v_i^0, w_i^0, r_i^0) = (S, 0, 0) \quad \forall i \in P
$$

Стоимость найма $x$ юнитов типа $c$ на старте вычисляется по формуле:
$$
p_c^0(x) = \sum_{k=1}^x k = \frac{x(x + 1)}{2}, \quad c \in \{w, r\}
$$
где $w$ обозначает воинов, а $r$ --- рабочих.

Каждый игрок выбирает пару найма:
$$
\text{hire}_i^0 = (w,r) \in D^0_{hire} = \{ (w,r) \in \mathbb{Z}_{\ge 0}^2 \ |\ p_w^0(w) + p_r^0(r) \le S \}
$$
где $w$ --- количество нанимаемых воинов, $r$ --- количество нанимаемых рабочих.

После покупки стартовых юнитов формируется начальное состояние для основного игрового процесса:
$$
\text{state}_i^1 = (S - (p_w^0(hire_{i,1}) + p_r^0(hire_{i,2})),\ hire_{i,1},\ hire_{i,2})
$$

Игра начинается с хода $t = 1$ в состоянии:
$$
G^1 = (\text{state}_1^1, \text{state}_2^1, \dots, \text{state}_N^1)
$$

\end{document}