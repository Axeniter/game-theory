\documentclass{article}
\usepackage[utf8]{inputenc}
\usepackage[T2A]{fontenc}
\usepackage[english, russian]{babel}
\usepackage{amsmath}
\usepackage{amsfonts}
\usepackage{amssymb}
\usepackage{parskip}
\usepackage{algorithm}
\usepackage{algpseudocode}
\setlength{\parskip}{1em}
\setlength{\parindent}{0pt}

\begin{document}

\section{Концепция игры}
\subsection{Завязка}
Раз в 100 лет в далеком королевстве проходит масштабное мероприятие,
великий аукцион, в котором разыгрывается нажитое за век богатство страны.
В это время со всего мира в столицу съезжаются авантюристы, путешественники,
торговцы и другие сомнительные личности, для того, чтобы сразится за право
участвовать в аукционе.
\subsection{Игра}
Аукцион века --- это бескоаллиционная неантагонистическая стратегическая игра для
$n$-го числа игроков, с полной информацией о параметрах и одновременным совершением ходов.
\subsubsection{Цель}
Цель каждого игрока максимизировать свои деньги, поучаствовав в аукционе.
\subsubsection{Начало}
Каждому игроку выдаётся $M$ монет, на которые они могут вести торги. На аукционе 
выставляется $L$ лотов, каждый лот имеет свою настоящую стоимость и участникам она
известна.
\subsubsection{Торги}
Каждый игрок распределяет свои ставки на лоты. Ставки делаются безвозвратно, тоесть 
игрок тратит поставленные монеты независимо от исхода лота 
Он имеет право оставить некоторое
число монет при себе, не ставя их, соответственно он может вообще ничего не поставить.
После того как ставки сделаны, выбирается победитель для каждого лота. Лот забирает
тот, чья ставка на него оказалась самой большой. Если максимальных ставок несколько,
то лот получает случайный игрок из тех, кто поставил максимальную ставку.
После торгов высчитывается баланс каждого участника.
\subsection{Обоснование принадлежности к бескоаллиционным неантагонистическим играм}
\subsubsection{Бескоаллиционность}
В бескоалиционных играх участники действуют независимо, не заключая формальных или
обязательных соглашений о совместных действиях, распределении выигрышей и т.д.

В данной игре игроки распределяют свои ставки индивидуально, скрытно и одновременно.
Правила не предусматривают возможности образования коалиций с общим фондом ставок 
или обязательством делиться выигранными лотами. Результат для каждого игрока зависит
только от его собственных ставок и ставок остальных, без механизмов кооперативного 
перераспределения внутри игры. Даже если несколько игроков случайно поделят лот
при равенстве ставок, это не является результатом предварительной договорённости, 
а следует из правил случайного распределения.

Таким образом, игра является бескоалиционной.
\subsubsection{Неантагонистичность}
Антагонистической (или игрой с нулевой суммой) называется игра, в которой сумма
выигрышей всех игроков постоянна (часто равна нулю). В неантагонистических играх 
сумма выигрышей может меняться, и интересы игроков не обязательно противоположны.

Это не игра с нулевой суммой, потому что интересы игроков не являются 
строго противоположными: возможно, что все игроки получат положительный 
ожидаемый выигрыш относительно своих ставок или стратегий, либо 
несколько игроков одновременно окажутся в выигрыше или проигрыше.

Следовательно, игра не является игрой с нулевой суммой, 
то есть относится к неантагонистическим.

\section{Математическая формализация}

\subsection{Множество игроков}
Пусть 
$$
N = \{1, 2, \dots, n\}, \quad n \geq 2
$$
конечное множество игроков.

\subsection{Параметры игры}
\begin{itemize}
    \item $M \in \mathbb{N} > 0$ --- начальный капитал каждого игрока (в монетах).
    \item $L \in \mathbb{N} > 0$ --- количество лотов.
    \item Для каждого лота $\ell = 1, \dots, L$ задана его настоящая стоимость $v_\ell > 0$, известная всем игрокам.
\end{itemize}

\subsection{Множества стратегий}
Стратегией игрока $i \in N$ называется вектор ставок на все лоты. Формально:
$$
X_i = \left\{ 
    \mathbf{x}_i = (x_{i1}, x_{i2}, \dots, x_{iL}) \in \mathbb{N}^L_{\geq 0} 
    \ \Big|\ 
    \sum_{\ell=1}^{L} x_{i\ell} \leq M 
\right\}.
$$
Здесь:
\begin{itemize}
    \item $x_{i\ell}$ --- ставка игрока $i$ на лот $\ell$,
    \item условие $\sum_{\ell=1}^{L} x_{i\ell} \leq M$ означает,
    что затраты игрока не могут привысить его баланс, а также
    что игрок может сохранить часть денег.
\end{itemize}
$X_i$ является конечным множеством в $\mathbb{N}^L$, ограниченным бюджетным ограничением.

\subsection{Определение победителей лота}
Для фиксированного набора стратегий $(\mathbf{x}_1, \dots, \mathbf{x}_n)$ и для каждого лота $\ell$ определим:
\[
m_\ell = \max_{j \in N} x_{j\ell} \quad \text{(максимальная ставка на лот $\ell$)}.
\]
\[
W_\ell = \{ i \in N \mid x_{i\ell} = m_\ell \}
\]
--- множество игроков, сделавших максимальную ставку на лот $\ell$.

Правило распределения лота:
\begin{itemize}
    \item Если $|W_\ell| = 1$, то лот полностью получает единственный победитель.
    \item Если $|W_\ell| = k > 1$, то лот целиком достаётся одному случайно выбранному 
    игроку из $W_\ell$ с равной вероятностью $1/k$ для каждого.
\end{itemize}

\subsection{Функции выигрыша}
Введём случайную величину $r_{i\ell}$ --- доля лота $\ell$, полученная игроком $i$:
$$
r_{i\ell} = 
\begin{cases}
1, & \text{если игрок $i$ выиграл лот $\ell$}, \\
0, & \text{в противном случае}.
\end{cases}
$$

При $|W_\ell| = k > 1$ имеем вероятность выигрыша лота:
$$
\mathbb{P}(r_{i\ell} = 1) = 
\begin{cases}
\frac{1}{k}, & \text{если } i \in W_\ell, \\
0, & \text{если } i \notin W_\ell.
\end{cases}
$$

Тогда выигрыш (баланс) игрока $i$ в денежном выражении:
$$
u_i(\mathbf{x}_1, \dots, \mathbf{x}_n) = 
\underbrace{M - \sum_{\ell=1}^{L} x_{i\ell}}_{\text{остаток после ставок}} 
+ \sum_{\ell=1}^{L} r_{i\ell} \cdot v_\ell.
$$

\subsection{Ожидаемый выигрыш}
Рассмотрим математическое ожидание выигрыша:
$$
\mathbb{E}[u_i(\mathbf{x}_1, \dots, \mathbf{x}_n)] = 
M - \sum_{\ell=1}^{L} x_{i\ell} 
+ \sum_{\ell=1}^{L} \mathbb{E}[r_{i\ell}] \cdot v_\ell,
$$
где
$$
\mathbb{E}[r_{i\ell}] = 
\begin{cases}
1, & \text{если } |W_\ell| = 1 \text{ и } i \in W_\ell, \\
\frac{1}{|W_\ell|}, & \text{если } |W_\ell| > 1 \text{ и } i \in W_\ell, \\
0, & \text{если } i \notin W_\ell.
\end{cases}
$$

\subsection{Нормальная форма игры}
Игра задаётся кортежем:
$$
\Gamma = \big\langle N, \{X_i\}_{i \in N}, \{u_i\}_{i \in N} \big\rangle.
$$

\end{document}