\documentclass{article}
\usepackage[utf8]{inputenc}
\usepackage[T2A]{fontenc}
\usepackage[english, russian]{babel}
\usepackage{amsmath}
\usepackage{amsthm}
\usepackage{amsfonts}
\usepackage{amssymb}
\usepackage{parskip}
\usepackage{algorithm}
\usepackage{algpseudocode}
\setlength{\parskip}{1em}
\setlength{\parindent}{0pt}

\begin{document}

\section{Концепция игры}
\subsection{Завязка}
Раз в 100 лет в далеком королевстве проходит масштабное мероприятие,
великий аукцион, в котором разыгрывается нажитое за век богатство страны.
В это время со всего мира в столицу съезжаются авантюристы, путешественники,
торговцы и другие сомнительные личности, для того, чтобы сразится за право
участвовать в аукционе.
\subsection{Игра}
Аукцион века --- это бескоаллиционная неантагонистическая стратегическая игра для
$n$-го числа игроков, с полной информацией о параметрах и одновременным совершением ходов.
\subsubsection{Цель}
Цель каждого игрока максимизировать свои деньги, поучаствовав в аукционе.
\subsubsection{Начало}
Каждому игроку выдаётся $M$ монет, на которые они могут вести торги. На аукционе 
выставляется $L$ лотов, каждый лот имеет свою настоящую стоимость и участникам она
известна.
\subsubsection{Торги}
Каждый игрок распределяет свои ставки на лоты. Ставки делаются безвозвратно, тоесть 
игрок тратит поставленные монеты независимо от исхода лота 
Он имеет право оставить некоторое
число монет при себе, не ставя их, соответственно он может вообще ничего не поставить.
После того как ставки сделаны, выбирается победитель для каждого лота. Лот забирает
тот, чья ставка на него оказалась самой большой. Если максимальных ставок несколько,
то лот получает случайный игрок из тех, кто поставил максимальную ставку.
Если максимальная ставка на лот равна 0, тоесть никто на него не поставил,
то лот получает случайный участник. После торгов высчитывается баланс каждого участника.
\subsection{Обоснование принадлежности к бескоаллиционным неантагонистическим играм}
\subsubsection{Бескоаллиционность}
В бескоалиционных играх участники действуют независимо, не заключая формальных или
обязательных соглашений о совместных действиях, распределении выигрышей и т.д.

В данной игре игроки распределяют свои ставки индивидуально, скрытно и одновременно.
Правила не предусматривают возможности образования коалиций с общим фондом ставок 
или обязательством делиться выигранными лотами. Результат для каждого игрока зависит
только от его собственных ставок и ставок остальных, без механизмов кооперативного 
перераспределения внутри игры. Даже если несколько игроков случайно поделят лот
при равенстве ставок, это не является результатом предварительной договорённости, 
а следует из правил случайного распределения.

Таким образом, игра является бескоалиционной.
\subsubsection{Неантагонистичность}
Антагонистической (или игрой с нулевой суммой) называется игра, в которой сумма
выигрышей всех игроков постоянна (часто равна нулю). В неантагонистических играх 
сумма выигрышей может меняться, и интересы игроков не обязательно противоположны.

Это не игра с нулевой суммой, потому что интересы игроков не являются 
строго противоположными: возможно, что все игроки получат положительный 
ожидаемый выигрыш относительно своих ставок или стратегий, либо 
несколько игроков одновременно окажутся в выигрыше или проигрыше.

Следовательно, игра не является игрой с нулевой суммой, 
то есть относится к неантагонистическим.

\section{Математическая формализация}

\subsection{Множество игроков}
Пусть 
$$
N = \{1, 2, \dots, n\}, \quad n \geq 2
$$
конечное множество игроков.

\subsection{Параметры игры}
\begin{itemize}
    \item $M \in \mathbb{N} > 0$ --- начальный капитал каждого игрока (в монетах).
    \item $L \in \mathbb{N} > 0$ --- количество лотов.
    \item Для каждого лота $\ell = 1, \dots, L$ задана его настоящая стоимость $v_\ell > 0$, известная всем игрокам.
\end{itemize}

\subsection{Множества стратегий}
Стратегией игрока $i \in N$ называется вектор ставок на все лоты. Формально:
$$
X_i = \left\{ 
    \mathbf{x}_i = (x_{i1}, x_{i2}, \dots, x_{iL}) \in \mathbb{N}^L_{\geq 0} 
    \ \Big|\ 
    \sum_{\ell=1}^{L} x_{i\ell} \leq M 
\right\}.
$$
Здесь:
\begin{itemize}
    \item $x_{i\ell}$ --- ставка игрока $i$ на лот $\ell$,
    \item условие $\sum_{\ell=1}^{L} x_{i\ell} \leq M$ означает,
    что затраты игрока не могут привысить его баланс, а также
    что игрок может сохранить часть денег.
\end{itemize}
$X_i$ является конечным множеством в $\mathbb{N}^L$, ограниченным бюджетным ограничением.

\subsection{Определение победителей лота}
Для фиксированного набора стратегий $(\mathbf{x}_1, \dots, \mathbf{x}_n)$ и для каждого лота $\ell$ определим:
$$
m_\ell = \max_{j \in N} x_{j\ell} \quad \text{(максимальная ставка на лот $\ell$)}.
$$
$$
W_\ell = \{ i \in N \mid x_{i\ell} = m_\ell \}
$$
--- множество игроков, сделавших максимальную ставку на лот $\ell$.

Правило распределения лота:
\begin{itemize}
    \item Если $|W_\ell| = 1$, то лот полностью получает единственный победитель.
    \item Если $|W_\ell| = k > 1$, то лот целиком достаётся одному случайно выбранному 
    игроку из $W_\ell$ с равной вероятностью $1/k$ для каждого.
\end{itemize}

\subsection{Функции выигрыша}
Введём случайную величину $r_{i\ell}$ --- доля лота $\ell$, полученная игроком $i$:
$$
r_{i\ell} = 
\begin{cases}
1, & \text{если игрок $i$ выиграл лот $\ell$}, \\
0, & \text{в противном случае}.
\end{cases}
$$

При $|W_\ell| = k > 1$ имеем вероятность выигрыша лота:
$$
\mathbb{P}(r_{i\ell} = 1) = 
\begin{cases}
\frac{1}{k}, & \text{если } i \in W_\ell, \\
0, & \text{если } i \notin W_\ell.
\end{cases}
$$

Тогда выигрыш (баланс) игрока $i$ в денежном выражении:
$$
u_i(\mathbf{x}_1, \dots, \mathbf{x}_n) = 
\underbrace{M - \sum_{\ell=1}^{L} x_{i\ell}}_{\text{остаток после ставок}} 
+ \sum_{\ell=1}^{L} r_{i\ell} \cdot v_\ell.
$$

\subsection{Ожидаемый выигрыш}
Рассмотрим математическое ожидание выигрыша:
$$
\mathbb{E}[u_i(\mathbf{x}_1, \dots, \mathbf{x}_n)] = 
M - \sum_{\ell=1}^{L} x_{i\ell} 
+ \sum_{\ell=1}^{L} \mathbb{E}[r_{i\ell}] \cdot v_\ell,
$$
где
$$
\mathbb{E}[r_{i\ell}] = 
\begin{cases}
1, & \text{если } |W_\ell| = 1 \text{ и } i \in W_\ell, \\
\frac{1}{|W_\ell|}, & \text{если } |W_\ell| > 1 \text{ и } i \in W_\ell, \\
0, & \text{если } i \notin W_\ell.
\end{cases}
$$

\subsection{Нормальная форма игры}
Игра задаётся кортежем:
$$
\Gamma = \big\langle N, \{X_i\}_{i \in N}, \{u_i\}_{i \in N} \big\rangle.
$$

\section{Анализ игры}
\subsection{Доминирующие и доминируемые стратегии}

Рассмотрим свойства стратегий в данной игре. Стратегия $\mathbf{x}_i$ строго доминирует
стратегию $\mathbf{x}'_i$, если для любого набора стратегий остальных игроков $\mathbf{x}_{-i}$ выполняется:
$$
u_i(\mathbf{x}_i, \mathbf{x}_{-i}) > u_i(\mathbf{x}'_i, \mathbf{x}_{-i}).
$$

Слабо доминирует --- если неравенство нестрогое ($\geq$) и хотя бы для одного $\mathbf{x}_{-i}$ строгое.

Анализ стратегий:
\begin{enumerate}
    \item \textbf{Сохранение всех денег (нулевая стратегия)}. Рассмотрим стратегию 
    $\mathbf{x}^0_i = (0, 0, \dots, 0)$, когда игрок не делает ставок.
    Эта стратегия не доминируется стратегией с любыми положительными ставками.
    Действительно, если все остальные игроки делают очень высокие ставки на все лоты,
    то игрок $i$ с $\mathbf{x}^0_i$ получает выигрыш $M$, а при любой положительной ставке 
    он потратит хотя бы одну монету и с высокой вероятностью не выиграет лот, получив 
    $M-1$ или меньше. Следовательно, сохранение всех денег может быть оптимальным ответом
    в некоторых ситуациях.

    \item \textbf{Ставки выше стоимости лота}. Рассмотрим лот $\ell$ с настоящей стоимостью $v_\ell$. 
    Любая стратегия, в которой $x_{i\ell} > v_\ell$ (ставка игрока на этот лот превышает его стоимость), 
    строго доминируется любой стратегией, которая отличается тем, что на лот $\ell$ ставится 0,
    а сэкономленные деньги либо оставить у себя, либо использовать для ставок на другие лоты.
    \begin{proof}
        Пусть $\mathbf{x}_i$ содержит $x_{i\ell} > v_\ell$. Построим $\mathbf{x}'_i$, 
        где $x'_{i\ell} = 0$, а на остальные лоты ставки такие же.
        Тогда для любого $\mathbf{x}_{-i}$:
        \begin{itemize}
            \item Если игрок $i$ выигрывает лот $\ell$ при $\mathbf{x}_i$, то его выигрыш
            увеличивается на $v_\ell$, но он тратит $x_{i\ell} > v_\ell$, так что чистый
            убыток от победы на этом лоте составляет $x_{i\ell} - v_\ell > 0$.
            При $\mathbf{x}'_i$ он либо не выигрывает лот (и не тратит $x_{i\ell}$),
            либо выигрывает с вероятностью $1/n$. В любом случае, выигрыш при $\mathbf{x}'_i$
            будет строго больше.
            \item Если игрок $i$ не выигрывает лот $\ell$ при $\mathbf{x}_i$, то он просто
            теряет $x_{i\ell}$, что строго хуже, чем не тратить эти деньги при $\mathbf{x}'_i$.
        \end{itemize}
        Следовательно, ставки выше стоимости лота строго доминируемы.
    \end{proof}

    \item \textbf{Рациональное ограничение}. Из предыдущего пункта следует, что
    в любой недоминируемой стратегии должно выполняться:
    $$
    0 \leq x_{i\ell} \leq v_\ell \quad \text{для всех } \ell = 1,\dots,L.
    $$
    Таким образом, пространство стратегий можно сузить до:
    $$
    X_i^{\text{nd}} = \left\{ 
        \mathbf{x}_i \in \mathbb{N}^L_{\geq 0} 
        \ \Big|\ 
        \sum_{\ell=1}^{L} x_{i\ell} \leq M,\ 
        x_{i\ell} \leq v_\ell\ (\forall \ell)
    \right\}.
    $$

    \item \textbf{Полная трата бюджета не всегда оптимальна}. Стратегия, где 
    $\sum_{\ell} x_{i\ell} = M$, не доминирует стратегию с $\sum_{\ell} x_{i\ell} < M$.
    Сохранение части денег может быть выгодным, если ставки других игроков высоки,
    и шансы выиграть лот малы.
\end{enumerate}

\subsection{Равновесие Нэша}
\subsubsection{В чистых стратегиях}

Равновесие Нэша в чистых стратегиях --- такой набор $(\mathbf{x}_1^*, \dots, \mathbf{x}_n^*)$, что для каждого игрока $i$ и любой его альтернативной стратегии $\mathbf{x}_i$ выполняется:
$$
u_i(\mathbf{x}_i^*, \mathbf{x}_{-i}^*) \geq u_i(\mathbf{x}_i, \mathbf{x}_{-i}^*).
$$

Существование и свойства:
\begin{enumerate}
    \item \textbf{Тривиальное равновесие}. Набор стратегий, где все игроки выбирают 
    $\mathbf{x}_i^* = (0,0,\dots,0)$, является равновесием Нэша только если
    $v_\ell = 1$ для всех лотов $\ell = 1,\dots,L$.

    \begin{proof}
    Рассмотрим два случая:

    \textbf{Случай 1: $v_\ell > 1$ для некоторого $\ell$} \\
    Пусть игрок $i$ отклоняется, ставя 1 на лот $\ell$ (остальные ставят 0):
    \begin{align*}
    u_i(0,0,\dots,0) &= M \\
    u_i(\text{отклонение}) &= M - 1 + v_\ell > M
    \end{align*}
    Отклонение выгодно, значит $(0,0,\dots,0)$ не равновесие.

    \textbf{Случай 2: $v_\ell = 1$ для всех $\ell$} \\
    Пусть игрок $i$ отклоняется, ставя $x_{i\ell} > 0$ на лот $\ell$:
    \begin{align*}
    u_i(0,0,\dots,0) &= M \\
    u_i(\text{отклонение}) &= M - x_{i\ell} + v_\ell \leq M - 1 + 1 = M
    \end{align*}
    Отклонение не даёт дополнительной прибыли, значит $(0,0,\dots,0)$ равновесие.
    \end{proof}

    \item \textbf{Равновесия с низкими ставками}. При $n \geq 2$ и $v_\ell > 1$ могут существовать
    равновесия, где на некоторые лоты делаются минимальные ненулевые ставки.
    Например, рассмотрим симметричное равновесие для случая $L=1$, $v_1 = V > 1$, $M \geq 1$.
    \begin{itemize}
        \item Пусть все игроки ставят 1. Тогда выигрыш каждого (в ожидании):
        $$
        \mathbb{E}[u_i] = M - 1 + \frac{V}{n}.
        $$
        Если игрок отклоняется и ставит 0, его выигрыш: $M$ (так как остальные ставят 1,
        и он точно не выигрывает). Отклонение невыгодно, если:
        $$
        M - 1 + \frac{V}{n} \geq M \quad \Rightarrow \quad \frac{V}{n} \geq 1.
        $$
        Если $V \geq n$, то такое равновесие существует.
        \item Если же игрок отклоняется и ставит 2, то он выигрывает лот наверняка (при условии,
        что остальные ставят 1), получая $M - 2 + V$. Это отклонение невыгодно, если:
        $$
        M - 1 + \frac{V}{n} \geq M - 2 + V \quad \Rightarrow \quad 1 \geq V\left(1 - \frac{1}{n}\right).
        $$
        Это неравенство выполняется редко (только при очень малых $V$).
    \end{itemize}
    Таким образом, равновесие в чистых стратегиях существует не всегда и зависит от соотношения
    $v_\ell$, $n$ и $M$.

    \item \textbf{Общий случай}. В общем случае при $L > 1$ и различных $v_\ell$ равновесия
    в чистых стратегиях могут быть достаточно сложными. Часто возникают ситуации, когда
    ни один игрок не хочет увеличивать ставку на лот, потому что это потребует затрат,
    а вероятность выигрыша увеличится незначительно из-за конкуренции.
\end{enumerate}

\subsubsection{В смешанных стратегиях}

Поскольку игра конечная (стратегии ограничены бюджетом $M$ и условиями $x_{i\ell} \leq v_\ell$), 
то по теореме Нэша существует хотя бы одно равновесие Нэша в смешанных стратегиях.

\textbf{Свойства смешанных равновесий:}
\begin{enumerate}
    \item В смешанном равновесии игроки случайным образом выбирают ставки из множества
    недоминируемых стратегий $X_i^{\text{nd}}$.
    \item Для лота $\ell$ в смешанном равновесии должно выполняться условие безразличия:
    если игрок $i$ с ненулевой вероятностью ставит на лот $\ell$ некоторую сумму $s$,
    то его ожидаемая полезность от этой ставки должна быть равна ожидаемой полезности
    от других действий, доступных в поддержке его смешанной стратегии.
    \item В симметричном случае (все игроки одинаковы) можно искать симметричное равновесие,
    где все игроки используют одинаковое смешанное распределение.
\end{enumerate}

\textbf{Пример смешанного равновесия для простейшего случая:}
Пусть $n=2$, $L=1$, $v_1 = V > 1$, $M \geq V$.
Рассмотрим симметричное смешанное равновесие, где каждый игрок выбирает ставку
из множества $\{0, 1, 2, \dots, V\}$ с некоторыми вероятностями $p_0, p_1, \dots, p_V$.

В смешанном равновесии Нэша каждый игрок должен быть безразличен между всеми ставками
из поддержки своего смешанного распределения. Это приводит к системе уравнений,
определяющей вероятности $p_s$.

\textbf{Анализ условий безразличия:}
Пусть $F(s)$ --- вероятность того, что оппонент сделает ставку меньше $s$.
Тогда ожидаемый выигрыш игрока, делающего ставку $s$, равен:
$$
\mathbb{E}[u_i] = M - s + V \cdot F(s-1) + \frac{V}{2} \cdot \mathbb{P}(\text{оппонент ставит } s).
$$

В симметричном равновесии ставки $s$ и $s'$, принадлежащие поддержке распределения,
должны приносить одинаковый ожидаемый выигрыш. Решая соответствующие уравнения,
можно получить распределение вероятностей.

\textbf{Классический результат для аукционов:}
В аукционах с двумя игроками и дискретными ставками
существует симметричное равновесие в смешанных стратегиях, где:
\begin{itemize}
    \item Оба игрока рандомизируют свои ставки на некотором интервале $[0, b]$
    \item Распределение является непрерывным и строго возрастающим
    \item Ни один игрок не делает ставку выше стоимости лота
\end{itemize}

\textbf{Общий случай $L > 1$:}
При наличии нескольких лотов анализ смешанных равновесий значительно усложняется,
поскольку игроки распределяют бюджет между лотами. Однако теоретически
равновесие существует, хотя его явное построение может быть затруднительным.

\subsection{Выводы}
\begin{enumerate}
    \item \textbf{Доминируемые стратегии}: Ставки выше стоимости лота строго доминируемы,
    поэтому рациональные игроки никогда их не используют.
    
    \item \textbf{Пространство недоминируемых стратегий}: Ограничено условиями
    $0 \leq x_{i\ell} \leq v_\ell$ и $\sum_{\ell=1}^L x_{i\ell} \leq M$.
    
    \item \textbf{Равновесие в чистых стратегиях}: Существует не всегда,
    зависит от параметров игры. При $v_\ell = 1$ для всех лотов
    набор нулевых ставок является равновесием.
    
    \item \textbf{Равновесие в смешанных стратегиях}: Существует всегда
    (по теореме Нэша), хотя может быть сложным для построения.
    В простейших случаях соответствует рандомизации ставок.
\end{enumerate}
\end{document}