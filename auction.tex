\documentclass{article}
\usepackage[utf8]{inputenc}
\usepackage[T2A]{fontenc}
\usepackage[english, russian]{babel}
\usepackage{amsmath}
\usepackage{amsfonts}
\usepackage{amssymb}
\usepackage{parskip}
\usepackage{algorithm}
\usepackage{algpseudocode}
\setlength{\parskip}{1em}
\setlength{\parindent}{0pt}

\begin{document}

\section{Концепция игры}
\subsection{Завязка}
Раз в 100 лет в далеком королевстве проходит масштабное мероприятие,
великий аукцион, в котором разыгрывается нажитое за век богатство страны.
В это время со всего мира в столицу съезжаются авантюристы, путешественники,
торговцы и другие сомнительные личности, для того, чтобы сразится за право
учавствовать в аукционе.
\subsection{Игра}
Аукцион века - это бескоаллиционная неантагонистическая стратегическая игра для
$N$-го числа игроков, с полной информацией и одновременным совершением ходов.
\subsubsection{Цель}
Цель каждого игрока максимизировать свои деньги, поучавствовав в аукционе.
\subsubsection{Начало}
Каждому игроку выдаётся $M$ монет, на которые они могут вести торги. На аукционе 
выставляется $L$ лотов, каждый лот имеет свою настоящую стоимость и участникам
известно о ней.
\subsubsection{Торги}
Каждый игрок распределяет свои ставки на лоты. Он имеет право оставить некоторое
число монет при себе, не ставя их, соответственно он может вообще ничего не поставить.
После того как ставки сделаны, выбирается победитель для каждого лота. Лот забирает
тот, чья ставка на него оказалась самой большой. Если максимальных ставок несколько,
то лот распределяется случайно между игроками, которые поставили эти ставки.
После торгов высчитывается баланс каждого участника.
\subsection{Обоснование принадлежности к бескоаллиционным неантагонистическим играм}
\subsubsection{Бескоаллиционность}
В бескоалиционных играх участники действуют независимо, не заключая формальных или
обязательных соглашений о совместных действиях, распределении выигрышей и т.д.

В данной игре игроки распределяют свои ставки индивидуально, скрытно и одновременно.
Правила не предусматривают возможности образования коалиций с общим фондом ставок 
или обязательством делиться выигранными лотами. Результат для каждого игрока зависит
только от его собственных ставок и ставок остальных, без механизмов кооперативного 
перераспределения внутри игры. Даже если несколько игроков случайно поделят лот
при равенстве ставок, это не является результатом предварительной договорённости, 
а следует из правил случайного распределения.

Таким образом, игра является бескоалиционной.
\subsubsection{Неантагонистичность}
Антагонистической (или игрой с нулевой суммой) называется игра, в которой сумма
выигрышей всех игроков постоянна (часто равна нулю). В неантагонистических играх 
сумма выигрышей может меняться, и интересы игроков не обязательно противоположны.

В данной игре каждый лот имеет свою настоящую стоимость. Если игрок получает лот 
за сумму меньше настоящей стоимости, он получает положительную разницу как прибыль.
Однако, если победитель переплачивает (ставка выше стоимости лота), его выигрыш может
стать отрицательным (убыток), а другие игроки при этом не обязательно компенсируют этот 
убыток ровно на такую же положительную сумму. Кроме того, часть денег может остаться у 
игроков (не поставленные монеты), и они не переходят другим участникам.
Итоговый баланс игроков складывается из начальных денег, минус потраченные на ставки, 
плюс стоимость полученных лотов. Общая сумма денег в системе после торгов не постоянна, 
потому что добавленная стоимость лотов увеличивает общий капитал.

Следовательно, игра не является игрой с нулевой суммой, 
то есть относится к неантагонистическим.
\end{document}